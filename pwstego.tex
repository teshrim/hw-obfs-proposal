%%%%%%%%%%%%%%%%%%%%%%%%%%%%%%%%%%%%%%%%%%%%%%%%%%%%%%%%%%%%%%%%%%%%%%%%%%%%%%%%
\subsection{Low-entropy Steganographic Security and Password-based DSE}
\label{sec:pwstego}
%\tnote{Need to massage to ensure we have proper relationship between DSE and
%Stego sorted out here and in previous section}
%\tnote{Remember to cite \cite{Cachin:Info_Theory_Stego} regarding DSE
%constructions, probably in last section is better.}
%
The previous discussion of distribution-sensitive encryption was
strongly focused on the real-world censorship circumvention
problem. Let us step back now to consider a broader perspective on
steganography and new opportunities for DSC to have positive impact.

%\paragraph{Stegonagraphy in practice.}  
A classic approach for steganography 
attempts to hide a conventionally encrypted message
in the low bits of pixel data in an image
(c.f.,~\cite{johnson1998exploring,cheddad2010digital,chandramouli2004image}).
We first observe that this can be viewed as an encrypt-then-DTE
construction, just like the DSE schemes discussed in the last section:
here the DTE scheme is parameterized by one or more images, and encodes by substituting
ciphertext bits into appropriate locations in order to create the
covertext. 
The security target for image-based steganography is most often ad hoc, essentially
that human observers cannot distinguish between the original image and
the one with ciphertext embedded. But
this target leads to weak schemes, as simple statistical tests or other
algorithmic approaches can easily detect the presence of an embedded
ciphertext~\cite{chandramouli2004image,provos2003hide}. 

Compounding this is the fact that,
as with conventional encryption, many practical uses of conventional 
steganographic tools rely upon human-memorizable passwords. 
This makes them vulnerable to offline,
brute-force attacks that not only confirm the existence of hidden content but
also \textit{reveal the password and message}, as suggested by Provos and
Honeyman~\cite{provos2003hide}. The procedure is as follows.
Given a possible covertext~$C$, run decryption with each of the
plausible passwords.  
Should none decrypt to properly formatted plaintexts then one can conclude correctly,
with high probability, that~$C$ is in fact a benign message.  Otherwise, $C$~is
indeed a covertext and the message and password are revealed. The format might
be correct ASCII encoding of text or the like.  In a chosen-message attack
setting, including the steganographic security definition formalized by Hopper
et al., the attacker knows not just the format, but the entire plaintext.

%\paragraph{Steganographic security with low-entropy keys.} 
A priori, it may seem that secure steganography with
low-entropy keys is impossible.  But in specific settings,
message distributions are known (or can be well estimated), a fact we will seek to 
leverage.  As a concrete example, and one that is also
potentially important in the censorship setting, we will consider
password-protected, steganographic key distribution.  Here, 
the goal is to transmit the public key of
an asymmetric encryption or signature scheme to another party, while
hiding the exchange.  In this case, the messages 
come from a clearly-defined distribution, such as an element sampled uniformly
from a group suitable for discrete-log-based schemes, or a uniformly sampled RSA modulus and public
exponent.\footnote{Interestingly, we expect that the latter will be very
challenging, in fact it may only be feasible with machinery such as cryptographic
obfuscation~\cite{garg2013candidate}. The reason is that it would seem to require
having a sampling routine that produces a RSA modulus without knowing the
factors.} Crucially, the coins used to sample from these
distributions should be unknown to parties external to the exchange protocol.
%In other contexts, we will investigate use of empirically educated
%estimates of the message distribution. \tsnote{This last sentence
%  fizzles out.}


\begin{task}
\label{task:pwstego-defs}
We will formalize security notions for steganographic systems when used with
low-entropy keys such as human passwords. 
\end{task}

Specifically, we will develop security notions of \textit{low-entropy steganographic
  security}, wherein covertexts must be indistinguishable from
non-steganographic ``benign'' channel messages, even when low-entropy keys are used.
The key to achieving such a notion, as we've just suggested, 
will be to target security when the challenge messages are 
drawn from a distribution with coins unknown to the attacker.  
We will explore settings in which the attacker and scheme designer
know the distribution, as well as (arguably more practical) 
settings in which they are only able to each obtain some number of samples drawn
from it. 


\paragraph{New steganographic constructions.}
We will also pursue the design of constructions that meet
this new goal of low-entropy steganographic security.  This will not be a simple matter of tweaking existing
schemes that are provably secure when given high-entropy keys.
For example, our preliminary work shows that the schemes from Hopper et
al.~\cite{Hopper:Provable_Stego} are vulnerable, in the low-entropy
key setting, to a brute-force attack similar to
the one described above, even when the underlying message is 
a uniform bit string.  (The weakness in these schemes is the use of redundancy to ensure low
probability of decryption errors.) 

\begin{task}
\label{task:attack-hopper}
We will show an attack against the low-entropy steganographic security of the
Hopper et al.~scheme for uniformly selected random messages.
\end{task}

\noindent
Our initial approach will be to extend the encrypt-then-DTE construction to
include the techniques from HE.  That is, we will first apply a DTE tailored
to the plaintext distribution, then apply a carefully selected 
password-based encryption scheme, and finally, to obtain the ciphertext, apply to the intermediate ciphertext a DTE tailored to the cover
distribution. We call this a \textit{DTE-encrypt-DTE} construction.  When the
plaintext and ciphertext distributions are amenable to building DTEs,
such a construction has the potential to provide steganographic
security (the attacker can't determine if it is an encryption or just a ``benign''
message), \emph{even} in the face of 
brute-force attacks like the one described above. The reason will be that for any password,
decryption of a message, whether it be benign or the output of DTE-encrypt-DTE,
will produce a 
plausible plaintext message.   DTE-encrypt-DTE is just one framework for
building low-entropy steganography (and DSE), and we expect that our
work will uncover others.

\begin{task}
\label{task:pwstego-approaches}
We will develop new approaches to low-entropy, and especially
password-based, steganography.
\end{task}

%
%\tnote{need to work through this more...}
%\tsnote{Made a pass through this section, integrating with earlier DSE
%  stuff. Not sure we need to say much more.}

