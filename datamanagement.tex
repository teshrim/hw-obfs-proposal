%-------------------------------------------------------------------------------------
\section*{Data Management Plan} 

Our proposed experiments, system development, and outreach activities will
produce a number of different data artifacts. As dissemination of these
artifacts is essential to the goal of broad impact that is central to this
proposal, we have carefully formulated distinct data management policies
according to artifact types and sensitives. We anticipate that our proposal
will result broadly in the following three data types:

\begin{enumerate}
\item[(1)] Data sets resulting from measurement studies and experiments;
\item[(2)] Open-source software; and
\item[(3)] Curriculum materials.
\end{enumerate}

\paragraph{Data stewards.} Each task supervisor will be a designated steward for
all data resulting from the task, assuming responsibility for the management of
the data and determining an appropriate classification (public, conditionally
sharable, or private) for particular data artifacts. Should a data steward be
unable to assume continuing responsibility for certain datasets, due to
departure from his institution or some other event, he will transfer stewardship
to a senior researcher at his university and
will furnish the instruction and documentation required for the new steward to
assume continuing responsibility.

\paragraph{Data-handling policies.} We will make our data artifacts available to
other researchers as well as the general public to the greatest possible extent,
as consistent with privacy considerations. Our approach will be to identify
datasets as {\em public} (P), {\em conditionally releasable} (S), {\em
confidential} (C), or {\em educational} (E). We specify the associated policies
for each below. We will apply policy (P), (S), (C), or (E) to category (1) data
as deemed appropriate by the associated data steward. We will apply policy (P)
to category (2) data and policy (E) to category (3) data by default. Our
data-handling policies are as follows:

\begin{itemize}

\item{\em Policy (P): Public data.} Public datasets will be those suitable for
posting online, e.g., data derived from public sources, or the outputs of
experiments (e.g., data, source code) that themselves do not involve any
privacy-sensitive data. Our policy will be to retain these data for five years
from the date of publication of any paper relying on the data. We will retain
data for a longer period of time if possible, giving explicit priority to the
goal of ensuring long-term scientific reproducibility. Public data will be made
available via a project website or a public cloud. Larger data sets that cannot
be disseminated by either such means will be stored locally and instructions
will be published for interested researchers and others to obtain access to the
data. We will adhere to a policy of releasing all source code resulting from the
proposal as open-source software under suitable nonrestrictive licenses, and
will make use of repositories, e.g., GitHub, that support this practice.

\item {\em Policy (S): Conditionally releasable data.} Some data artifacts
produced by our work will carry either temporary sharing limitations (e.g.,
individually requested moratoria on the release of personal data) or permanent
ones. We will retain such data for the same duration of time as specified in
policy (P). These data will not be made public, but stored locally with
appropriate access-control mechanisms to restrict both external and internal
access or in a cloud with protections that are suitable to the sensitivity of
the data, e.g., a HIPAA-compliant cloud. Should researchers or others submit
appropriate requests for data access, we will confirm that the request is
appropriate (e.g., under the aegis of IRB-approved work) and will determine a
practicable minimal-release strategy, specifically exploring time-limited and
sanitized data-sharing approaches, as well as whether data should be released
directly or through a query interface. We will release the data as expeditiously
as possible, consistent with resource and policy constraints.

\item {\em Policy (C): Confidential data.} A data steward may deem some data
temporarily or permanently unsuitable for release outside his institution.
%
\if{0}{
University network packet traces such as we expect to collect in the course of
this proposal will be deemed confidential in all cases, while derived data such
as non-personally identifiable aggregate statistics or anonymized packet headers
may be categorized as (S) or (P). Other data may additionally be deemed
confidential by the data steward.
}\fi
%
At the time of data collection, the steward
will determine whether it is appropriate to erase the data. (For example, highly
sensitive data not employed in research may be summarily deleted.) Otherwise,
the data will be preserved according to Policy (S), but with no access granted
outside the institution of the data steward.

\item {\em Policy (E): Educational data.} The data produced in curriculum
development in the context of this project will be handled under Policy (P).
These data will be made publicly accessible on the website of the data steward
or in an appropriately locatable and accessible public archive.  
\end{itemize}

\paragraph{Data storage and lifetime.} The volume of data produced in this
proposal will be small enough to permit handling within the existing data
storage facilities of our university. At a minimum, data will be
stored for the duration of the project. We anticipate storing most data for
considerably longer, however, and will store for as long
as is practical: both data required to reproduce published experiments and data
of public value. We will store all data in suitable standard formats and will
confirm that university facilities include access controls and encryption as
suitable for the handling of specific data artifacts.

\paragraph{Vulnerability disclosures.} This research project does not explicitly
encompass vulnerability assessments. It is very well possible, however, that we
will discover security vulnerabilities or inappropriate data disclosures in the
course of our work.
\if{0}{For example, in our statistical modeling of data we may
uncover inadvertent leakage of personally identifiable information; in our study
of censorship circumvention we may discover vulnerabilities that expose
confidential data to censors.
}\fi
%
We will adhere broadly to community-standard responsible disclosure practices.
Specifically, we will follow the following steps in disclosing a vulnerability:

\begin{enumerate}
\item {\em We will identify stakeholders.} We will identify primary stakeholders, entities developing or managing the affected systems or data, as well as secondary stakeholders, those potentially harmed by the vulnerability, e.g., users of the impacted system or subjects of the relevant data. We will work as advocates for secondary stakeholders throughout the disclosure process.

\item {\em We will privately disclose the vulnerability.} We will notify primary stakeholders of the vulnerability and provide tangible evidence so that they can confirm and assess its scope. We will seek to make this disclosure as expeditiously as possible.
\item {\em We will assist in vulnerability remediation.} We will advise primary stakeholders on technical remediation strategies, as appropriate.
\item {\em We will create a public disclosure plan.} In consonance with research community practice, we will by default make a public disclosure that specifically identifies the vulnerability, modifying this approach if it may bring about harm to secondary stakeholders and working with primary stakeholders to determine the appropriate level of detail to disclose about the vulnerability. Upon discovery of the vulnerability, we will set a target date for public disclosure. By default, this will be 90 days from private disclosure of the vulnerability.
\item {\em We will conduct a review with primary stakeholders.} We will circulate drafts of the public disclosure to primary stakeholders, soliciting their feedback and working with them to ensure that details are correct and amending the disclosure as appropriate, taking into account any harm that may affect primary and
secondary stakeholders as a result of disclosure.
\item {\em We will make a public disclosure of the vulnerability.} We will publish the disclosure, including both technical detail and explanations accessible to secondary stakeholders, as warranted by the vulnerability.
\end{enumerate}






