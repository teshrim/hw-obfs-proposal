\documentclass[11pt]{article}
\usepackage{times}
\usepackage{paralist} %compactitem
\sloppy
\pagestyle{empty}
\begin{document}
\section*{Facilities, Equipment, and Other Resources}

PI Ristenpart will be involved in all aspects of the research project for all
four years of the period of performance. This section highlights existing
equipment at the University of Wisconsin that will support this work. The
Budget Justification discusses the limited equipment requested in this
proposal.  Together, these sources provide ample resources for conducing the
proposed research and educational tasks.


\paragraph{Facilities and Equipment.} 
Existing UW equipment falls into two classes: (1) workstations
for software development, paper-authoring, email, etc.
(2) shared departmental resources that the PI benefits from and influences
but does not exclusively control.


\begin{compactenum}
\item {\bf Workstations.} The PI controls a variety of desktop and
laptop computers (approximately one desktop system and one laptop per
person). These will suffice for most experimentations. 
%These machines have a functional life of approximately three
%years. Funds are requested to provide a new workstation for each PI
%and RA, staggered over the course of the grant. 

%\item {\bf WiCT Testbed.} The PI controls
%a small laboratory testbed, including currently 10 server 
%machines and a switch, for experiments. 
%These include Intel processors matching those observed on Amazon EC2.

\item {\bf Shared Departmental Resources.} The PI and his students
will also have shared access to much of the computing resources of the
Computer Sciences Department. The Computer Sciences
Department operates the Computer Systems Laboratory, which supports
department computing for both research and instruction. The current
research computing equipment fits into four categories: workstations,
file servers, parallel computers, and networking. The Department
currently has approximately 550 desktop workstations, 125 servers, and
400 cluster nodes allocated for research computing. The Lab also
supports more than 24 terabytes of disk space mostly through
approximately 12 Advanced File System (AFS) servers. All of the
computers are connected through an Ethernet-based local area network
(desktop machines at 100Mb/s and servers with Gigabit Ethernet) with a
Gigabit Ethernet backbone, including Cisco 7000 routers. The network
connects via a 10 Gigabit Ethernet link to the UW-Madison campus
network, the Internet, and Internet II. 
\end{compactenum}

\end{document}
