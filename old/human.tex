%-------------------------------------------------------------------------------------
\section{Securing Human Secrets: Distribution-Sensitive Secure Sketches (DSSS)} 
\label{sec:human}

HE, DSE and password-based DSE treat the problems of reproducing
target distributions for messages, ciphertexts, and both
messages and ciphertexts, respectively.  We now consider the orthogonal problem of managing a
secret $\secret$ sampled from a target distribution. Of particular interest are
noisy, human-generated secrets such as  biometrics and (possibly mistyped) passwords. 
We consider the specific problem of deriving an error-detecting or
error-correcting value that helps deal with such noisy secrets, yet leaks as little
information as possible.
Applying the DSC framework to this challenge yields a new concept that
we call a {\em distribution-sensitive secure sketch} (DSSS). 

\paragraph{Problem setting.} Users often make typographical errors or
misremember ``something-you-know'' secrets, such as passwords and answers to
life questions. Similarly, biometrics, or ``something-you-are'' authentication
values, are inherently noisy; e.g., fingerprint images vary due to variability in pressure,
rotation, skin condition, and so forth.

Systems that manage user credentials securely in explicit form can correct errors
by explicit comparison. A server that stores a fingerprint template $\secret^*$ can
compare it against a fingerprint reading $\secret$ proffered by  a user at login,
accepting $\secret$ as valid if $\secret^*$ and $\secret$ are close under some suitable
metric. Conventional cryptographic primitives, though, are specifically designed
to be brittle in the face of noise and actually prevent direct comparison.
Flipping a single bit of input, for example, yields a completely different
output value when using a well-designed hash function. 

Such intolerance to noise or error often impedes secure and usable system
design. For instance, biometric templates {\em cannot be
protected via hashing}. Similarly, hashed passwords offer no {\em typo-safety}.
In hashed form, a valid password is indistinguishable from an incorrect one. In
honey encryption and password-based DSE, lack of typo-safety is especially
problematic: typos induce incorrect (but plausible-looking) decryptions that can
mislead legitimate users. 

\paragraph{Existing approaches.} {\em Fuzzy
cryptography}~\cite{Boy04,DKRS06,DORS06,DORS08,FJ01,Juels:2006fk,JuelsWattenberg:1999}
is one approach to reconciling conventional cryptography with noisy secrets. A
foundational tool is a {\em secure sketch} (the underpinning of a tool known as
{\em fuzzy extraction}). 

A secure sketch is a pair of efficient probabilistic algorithms $(\ssketch,
\rec)$ that operate over a secret space $\kspace$. The procedure $\ssketch$
takes an input secret $\secret \leftarrow_{\kdist}\kspace$, where $\kdist$ is a
probability distribution over $\kspace$; it outputs a ``helper'' string $s \in
\{0,1\}^*$. The procedure $\rec$ inputs a (possibly corrupted) secret $\secret'
\in \kspace$ and helper string $s$ and yields an output in $\kspace \,\cup\,
\bot$ (where $\bot$ indicates a decoding failure). An $(e, \tilde{e}, t)$-secure
sketch is one for which: (1) Recovery is possible for small errors, i.e.,
$\rec(\secret', \ssketch(\secret)) = \secret$ if $\dist(\secret, \secret') \leq t$ for some
distance metric $\dist$ and (2) The information leakage of the helper string in
terms of average min-entropy is small. Specifically, if  $\tilde{H}_{\infty}
(\kdist) \geq e$ then $\tilde{H}_{\infty} (\kdist \,|\, \ssketch(\kdist)) \geq
\tilde{e}$.

A secure sketch may be viewed as a systematic error-correcting code on 
$\secret$. What distinguishes it from an ordinary such code is the bound
(property (2)) on information leakage produced by the redundancy $s$ of the
code.

The security definition for a secure sketch is parameterized by the average
min-entropy of the secret distribution $\kdist$. But existing {\em secure sketch
constructions lack distributional sensitivity}. For example, Dodis et
al.~\cite{DORS08} describe a secure sketch for edit distance, a measure apt for,
e.g., typos in passwords. Their construction is agnostic to the underlying
distribution, and thus achieves only loose (and only asymptotically
bounded) security  
%$e' = e - t(\log F)2^{O\big(\sqrt{\log(n \log F) \log \log(n \log F)}\big)}$,
$e' = e - t\cdotsm(\log F )\cdotsm\exp(O\big(\sqrt{\log(n \log F) \log \log(n \log F)}\big))$, where $F$ is the alphabet size for characters
of strings in $\kspace$, and $n$ their length. 

As a simple example, a construction with $t=2$ on
eight-digit passwords over ASCII characters would have an average
min-entropy bound of at least $t \log F=16$ bits (excluding the exponential
asymptotic factor and its potentially large constants), but 
the average min-entropy of a typical password distribution in the wild~\cite{Bonneau12,Bonneau12b} is just under 7 bits!

Such drawbacks, as well as a general inability by the research community hitherto to match theory to real applications, has meant that secure sketches (and fuzzy extraction) have seen little or no use in practice.

\paragraph{Distribution-sensitive secure sketch (DSSS).} The DSC framework
points the way to a dramatic reduction in entropy loss that can bring fuzzy
cryptography into the realm of practicality. The result is DSSS, a secure sketch
that is engineered for a specific target secret distribution.

As a specific focus and example, we now consider a DSSS construction for
typographical errors in password entry that promises to improve greatly on the edit-distance secure sketch of Dodis et al.~for this application. The goal is a secure sketch
$(\ssketch, \rec)$ for a distribution $\kdist$ of user-selected passwords, i.e.,
over passwords $\secret \leftarrow_{\kdist} \kspace$, where $\kspace$ is the set of
permissible passwords in a system. The secure sketch should permit password
recovery given a user-typed password $\secret'$---possibly with some typos---when
$\dist(\secret, \secret') \leq t$. Here, $\dist$ is edit-distance, a standard metric
for typographical errors, and $t$ is a system parameter that would in practice
  be fairly small, e.g., $t = 2$ or $3$. 

\iffalse
As noted above, the edit-distance secure sketches outlined in~\cite{DORS08}
produce an average min-entropy entropy loss potentially greater than that of
real passwords. The reason is that these secure sketch constructions are
agnostic to underlying message distributions.
\fi

In the case of passwords, it is possible to leverage two key observations to achieve a practical DSSS. 
%\noindent\emph{Observation 1:} 
First, relatively large outputs 
are often acceptable in a password DSSS. For example, adding 10\,KB---or even 100\,KB---of
secure sketch data to protect a ten-character master password in a password
manager on a laptop is eminently practical, even though it creates a more than
1,000x or 10,000x expansion relative to the password itself. Thus it is possible
to base a secure sketch in such settings on an extremely low-rate
error-correcting code, potentially an unorthodox, application-specific one.
Second, popular user-selected passwords studied in the
wild have a high average pairwise edit distance. In order of popularity, here
are the top twenty passwords leaked from the RockYou data set that meet the
(broadly representative) password composition policy of iCloud:\vspace*{-.4em}
{\small
\begin{verbatim}
          Password1          Princess1          P@ssw0rd          Passw0rd          Michael1   
          Blink182           !QAZ2wsx           Charlie1          Anthony1          1qaz!QAZ  
          Brandon1           Jordan23           1qaz@WSX          Jessica1          Jasmine1
          Michelle1          Diamond1           Babygirl1         Iloveyou2         Matthew1
\end{verbatim}
}
\vspace*{-.4em}\noindent
One pair has edit distance $1$ ({\tt P@ssw0rd}, {\tt Passw0rd}) and another pair has edit distance $2$ ({\tt Password1}, {\tt Passw0rd}), but the vast majority have edit distance  at least $3$. Thus the set of frequently used passwords itself largely constitutes a {\em good codebook for edit-distance}. The construction in~\cite{DORS08} provides no way to make use of this essential insight.

To illustrate how these two observations can give rise to a practical DSS for
the special case of passwords, suppose that the complete set $\kspace$ of
passwords selected by a given population of users had minimum edit distance $4$.
Then a simple DSSS for $t = 2$ would be the pair $(\dsssketch, \dsrec)$, where
$\dsssketch(\secret) \rightarrow \kspace$ and $\dsrec(\secret')$ maps $\secret'$
to the closest $\secret \in \kspace$ by edit-distance. In practice, given the
major password leaks such as the RockYou breach, $\kspace$ may be effectively
viewed as public knowledge. Thus this DSSS has {\em no conditional entropy
loss}, i.e., $\tilde{H}_{\infty} (\kdist) -  \tilde{H}_{\infty} (\kdist \,|\,
\dsssketch(\kdist)) = 0$. In practice real password spaces will not have all
passwords as far apart as four, a challenge we discuss below.

Our two observations regarding passwords apply in other key settings (e.g.,
biometrics) and form an important foundation for exploration of DSSSs. They
illustrate respectively the two following general design principles:

\paragraph{\em DSSS Design Principle 1: Use of low-rate codes:} Unlike typical
settings for error-correcting codes, in many
settings of interest for DSSSs, it is feasible to use a very low-rate code,
i.e., a code with high redundancy. 

\paragraph{\em DSSS Design Principle 2: Use of embedded codebook structures:}
Target distributions of interest (e.g., biometrics and passwords) for DSSSs
typically have high average pairwise distance among high-probability secrets.
Sets of such secrets may be leveraged as codebooks to facilitate
error-correction / error-detection. We refer to such sets
of secrets as {\em embedded codebook structures}. 
This design principle gives rise to the following definitional retooling task:

\begin{task}
\label{ECS-task}
We will develop formal definitions and a general framework for identifying and quantifying the utility of embedded codebook structures in target message distributions for DSSS constructions.
\end{task}

Task~\ref{ECS-task} is challenging because of the messiness of real-world
message distributions. Recall in our example above that many password pairs will have low 
edit distance (e.g., ({\tt P@ssw0rd}, {\tt Passw0rd})), so not all
of $\kspace$ can serve as the codebook. 
A challenging optimization problem results: $\dsssketch$ must
output a subset of $\kspace$ that is pruned to eliminate low-distance pairs but
still encompasses as many high-weight, i.e., popular, passwords as possible. To
take a graph-theoretic perspective, consider a graph $G = (V,E)$ in which nodes
correspond to passwords in $\kspace$ and an edge $(\secret_i, \secret_j)$ is
present if $\dist(\secret_i, \secret_j) < t$. Intuitively, to yield a suitable
codebook, $\ssketch$ must output an independent set $\tilde{V} \subseteq V$; to preserve
conditional average min-entropy, $\tilde{V}$ must have close to maximal cumulative
weight under $\kdist$.  The maximal independent set problem is NP-hard, as  are
weighted variants, so practical solutions must rely on heuristics.

Another challenge is that the set $\kspace$ of possible user-selected passwords is
of vast size and not explicitly storable. A second challenge in constructing a
DSSS for this example application is handling relatively rare passwords that
might not appear in an explicitly stored password list. In general, we will
undertake the following task:

\begin{task}
\label{task:dsss-cons}
Building on DSSS Design Principles 1 and 2, we will use the DSC framework to explore the construction of DSSSs for a wide range of settings of interest, including passwords and biometrics. We will leverage knowledge obtained in specific applications to surface foundational concepts, such as embedded code structures.
\end{task}

Our DSC framework will guide empirical validation in this task. As the problems
underpinning our DSSS design approach are NP-hard, our constructions will rely
on approximation heuristics developed by the research community for related
problems, e.g.,~\cite{BermanFurer:1994,Halldorsson:2000}. Our analysis of our DSSS constructions will thus require validation of
these heuristics against real-world password data. Specifically, we will
experimentally measure the conditional entropy yielded in our DSSS construction
according to the formal security definitions for secure sketches.

\paragraph{DSSS for typo-safety in HE.} Because HE has the property that
decryption under an incorrect key yields a valid-looking plaintext, a problem
arises in the use of HE with noisy or error-prone keys / secrets (e.g.,
biometrics or user-typed passwords). An error in a decryption key can present a
legitimate user with an incorrect plaintext that the user may accept as valid. Despite promising proposed solutions, e.g., use of security skins~\cite{dhamija2005battle} and online testing for the case of password managers, this problem represents a major issue in the practical use of HE. DSSSs offer a potential solution.

In the setting of password-based HE, this issue of noisy secrets emerges specifically as a
problem of {\em typo safety}, namely the ability to detect a mistyped password
and trigger a decryption failure. It would be very helpful in practice to have a
mechanism that provides typo safety while {\em causing only minimal conditional
entropy degradation for an HE password}.

A DSSS for edit-distance over passwords provides exactly this property.
Additionally, in practical settings of interest, such as the SweetPass tool
described above, it suffices to detect rather than correct errors. Given a code with minimum distance $d$ in a DSSS construction (e.g., $\tilde{V}$ in the graph-based approach), error-correction can correct up to $t = \lfloor d/2 \rfloor$ errors in general. Error-detection, though, permits detection of a much improved $t = d-1$ errors using the same code.

\paragraph{Honeywords as DSSS.} DSSS also provides a new perspective on the honeywords system of Juels and Rivest~\cite{Honeywords:2013}. That system stores for each user a list of (hashed) passwords $s = \{\secret_1, \ldots, \secret_n\}$, of which $\secret_j$ is the true password for a random $j \leftarrow_R [1,n]$. The other passwords are fakes selected from a model distribution $\ddist$, i.e., $\secret_i \leftarrow_{\ddist} \kspace$ for $i \neq j$. The index $j$ is stored in an isolated system called a honeychecker. When a user proffers a password $K$ at authentication, if $\secret = \secret_i$ for $\secret_i \in s$, the index $i$ is sent to the honeychecker for verification. An adversary that breaches the system and learns $s$ still faces the challenge of guessing the right index $j$; a wrong guess signals the adversary's compromise of $s$.

The list $s$ may be viewed as the helper string in a DSSS $(\dsssketch, \dsrec)$ with $t=0$. An ideal password model, i.e., $\ddist = \mdist$, yields a $(e, \tilde{e}, 0)$-secure DSSS with $e - \tilde{e} = \log n$. (Indices may be replaced with values yielded by fuzzy extraction, to complete the conceptual connection.) If $\ddist$ poorly models $\mdist$, less security is obtained. This new DSSS-based perspective promises to yield innovations in honeywords construction. For instance, combining our ``programmable'' PRG with a model for sampling $\ddist$ yields a novel representation $s$ of {\em constant size for any $n$}.




